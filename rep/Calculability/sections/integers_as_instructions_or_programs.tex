\section{Integers as instructions or programs}

\subsection{Integer representation}
When we deal with RAM program, as said in Section~\ref{sec:RAM_intro}, we do not treat subtraction as if we were in $\mathbb{Z}$ or other set of number including $\mathbb{Z}$, we are supposed to be in the natural number world. 

In a RAM program subtraction is represented by the $\dotminus$ symbol giving us an expression of the type $A \dotminus B = max(0, A - B)$ therefore the we created the \textit{Int} class that extends the \textit{int} class and overrides the \textit{\_\_sub\_\_} function as shown in Listing \ref{listing:int_class_sub} .

\begin{lstlisting}[
    caption={The Int class}, 
    label={listing:int_class_sub},
    language=Python,
    backgroundcolor=\color{gray!10},
    frame=lines,
    ]
class Int(int):
    def __sub__(self, x: int):
        return Int(max(int(self) - x, 0))
    ...
\end{lstlisting}


\subsection{Gödelisation function}
As seen in course every instruction can be represented as an integer, thanks to Gödel bijective function (look at listing~\ref{god_enc}).

\begin{lstlisting}[
    caption={Gödel encoding function}, 
    label={god_enc},
    mathescape, 
    frame=lines,
    backgroundcolor=\color{gray!10}, breaklines=true
    ]
    $Rk =  Rk + 1 \rightsquigarrow 3 \times  k$ 
    $Rk =  Rk - 1 \rightsquigarrow 3 \times  k + 1$
    $IF\;Rk \neq  0\;THEN\;GOTOB\;n\;\rightsquigarrow 3 \times  [k, [1, [n, 0]]] + 2$ 
    $IF\;Rk \neq  0\;THEN\;GOTOF\;n \rightsquigarrow 3 \times  [k, [0, [n, 0]]] + 2$
 Where : 
    - k is the register number
    - is the number of jump to do inside a if
\end{lstlisting}

\newpage
\subsection{Gödel inverse function}
When we are given an integer and we know that it is an instruction, we can easily decode it with modulo operations (look at listing~\ref{god_dec}).
\begin{lstlisting}[
    caption={Gödel decoding function}, 
    label={god_dec},
    mathescape, 
    frame=lines,
    backgroundcolor=\color{gray!10},
    ]
if $X \equiv 0 \mod 3 \rightsquigarrow$ we have a $Rk = Rk + 1$ instruction
if $X \equiv 1 \mod 3 \rightsquigarrow$ we have a $Rk = Rk - 1$ instruction
  in both cases, k is equal to $\lfloor X/3 \rfloor$
if $X \equiv 2 \mod 3 \rightsquigarrow$ we have a jump instruction
  where $X' =  \lfloor X/3 \rfloor$ and we can decode $X$ in $[k, [j, [n, 0]]]$ 
  with Cantor inverse function (see next paragraph)
\end{lstlisting}
Decoding an instruction like can be done by the \textit{decode\_int\_instr} in \textit{decode\_int.py} file.

\subsection{Cantor pairing function}
\label{sec:cant_pair_func}
Now, we know how to code an instruction, we can so code a program, that is a list of instruction, thanks to Cantor function (look at listing~\ref{cant_enc})

\begin{lstlisting}[
    caption={Cantor encoding function}, 
    label={cant_enc},
    mathescape, 
    frame=lines,
    backgroundcolor=\color{gray!10},
    ]
    $[x, y] = \frac{(x + y) \times (x+y+1)}{2} + y + 1 = n$
    if b is a couple, we encode it at first,
    and that recursively until the last tuple
\end{lstlisting}

The encoding of two integer to a Cantor's int is made by the \textit{cantor} method of the \textit{Int} class.

\subsection{Cantor inverse function}
The Cantor inverse function is not as simple as computing the Gödel inverse one. The Cantor Pairing function is a \textit{Diophantine equation} (voir definition at ...) and finding back the two variables $x$ and $y$ from $n$ ask to iterate from 0 the n to find an intermediate $n'$.  

Since $n$ may be very big, (it is the case if we encode multiple instructions) this task may become very slow, therefore, to speed it, we make a \textit{binary search} to have an answer in a logarithmic time instead of a linear one. 

The decoding function of a Cantor's int to a couple of the form $(x_1, (x_2, (..., (x_n, 0))))$ is made by the \textit{Int} class with the \textit{int\_to\_couple} method.
\newpage
\begin{lstlisting}[
    caption={Cantor decoding function}, 
    label={cant_dec},
    frame=lines,
    backgroundcolor=\color{gray!10},
    language=Python
    ]
class Int(int):
    ...
    def cantor_inv(self):
        tmp = self.aux()
        r = self - ((tmp - 1) * tmp / 2 + 1)
        l = tmp - r - 1
        return l, r
    
    def int_to_couple(s):
        res = Int(s).cantor_inv()
        return res if res[1] == 0 
               else (res[0], Int(res[1]).int_to_couple())
    ...
\end{lstlisting}